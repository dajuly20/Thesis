

\renewcommand{\pthImg}{\pthUmsetzung/img}
\renewcommand{\pthDoc}{\pthUmsetzung/doc}

\subsection{C++ Library LibPiFace}
Wie im Vorhergenenden Abschnitt *REF* beschrieben, bildet die Erweiterungskarte PiFace Digital 2 *REF* die Grundlage für diese Bachelorarbeit. Im Lieferumfang befindet sich eine in C geschriebene Library inklusive eines Lauffähigen Tests, welche ebenfalls auf GitHub unter https://github.com/piface/libpifacedigital zu finden ist. Diese Library stützt sich wiederrum auf die Library https://github.com/piface/libmcp23s17 welche den verbauten SoC über SPI anspricht. Im Laufe der Arbeiten fiel jedoch auf, dass die C Library nicht alle benötigten Funktionen enthielt. Da der Quellcode vorlag, und die Lizensierung Veränderungen am Quellcode zulässt, lag die Überlegung nahe die benötigten Funktionen direkt in der Library zu ergänzen anstatt sie im eigentlichen Projekt unterzubringen. Weiterhin schien es auch ein erstrebenswertes Lernziel zu sein, das Erstellen und Übersetzen von statisch bzw. dynamisch gelinkten Bibliotheken kennenzulernen. Zuletzt schien es schlichtweg die Sauberste Lösung zu sein. Zunächst wurde angenommen, dass sich auch mehrere Hardwaremodule per SPI mit einem einzelnen Rasperry Pi verbinden lassen. Dies ist technisch auch möglich, so bieten die eingesetzten Boards die Möglichkeit über einen Jumper eine Hardwareadresse einzustellen. Der Hersteller bot auch die passende Hardware an, um mehrere Boards mit einem Rasperry Pi zu verbinden. Jedoch wurden diese scheinbar mangels Nachfrage aus dem Sortiment genommen. Obwohl eine Bastellösung es immer noch ermöglichen würde, ist der Aufwand hierfür sehr hoch und scheint unwirtschaftlich. Leider wurde bis zu dieser Erkentniss schon einiges an Energie darin investiert, mehrere Boards zu unterstüzen. Dies ist auch der Grund wieso ein Objektorientierter Ansatz in C++ gewählt wurde - eine Instanz für jedes Harwaremodul. Ein weiterer Grund war, dass die Anwendung ohne Caching nicht performant genug war. Das heißt die zeitliche Lücke zwischen einer Änderung an einem Eingang bis zu dessen Auswirkung am Ausgang war deutlich spürbar. Dafür wurden Methoden vorgesehen, um das Caching ein un auszuschalten - bei eingeschaltetem Caching verändern die Methoden um Bytes und Bits zu schreiben, lediglich den Wert einer Instanzvariable. Derzeit muss das leeren des Chaces explizit mittels Aufruf der Methode \textit{flush()} erfolgen. Über ein Automatisches verfahren wurde nachgedacht, jedoch erwies sich der manuelle Aufruf als einfacher. 

\subsection{Parsen von Logikausdrücken}
Das Ziel dieses Projektes ist es, dass die Ausgänge der Steuerung in Abhängigkeit von Eingängen wie physikalischen Eigängen oder zum Beispiel Timer-bausteinen ein beziehungsweise ausgeschaltet werden. Dafür ist in einer Textdatei für jeden Ausgang eine Zeile vorgesehen. Eine Zeile beginnt hierbei mit dem zu definierenden Ausgang, also zum Beispiel Ho1, worauf ein Gleichheitszeichen zu folgen hat. Der gesamte Ausdruck hinter dem Gleichheitszeichen wird zur Laufzeit des Programms durchlaufen, wobei jedes vorkommende Paar von [ und ] durch eine Null oder eine Eins ersetzt. Innerhalb der Klammern, findet sich gleich genau wie bei dem Bezeichner vor dem Gleichheitszeichen, die jeweilige Bezeichnung der Abhängigkeit. Lautet die Zeile also etwa Ho0= [Hi0] \& [Hi1] so wird die Komplette erste Klammer durch den Wert von Hi0 ersetzt, wärend die zweite Klammer  durch den Wert von Hi1 ersetzt wird. Daraus ergibt sich dann, vorausgesetzt Hi0 und Hi1 sind im Zustand \chphl{Ein}, Ho0= 1 \& 1. Das für den Leser offensichtliche Ergebnis dieser Gleichung ist 1 oder \chphl{true}. Jedoch gestaltet sich eine programmatische Lösung des Problems als deutlich komplexer. Denn sobald mehr als drei Ausdrücke im Spiel sind, müssen wie bei klassischer Mathematik Rechenregeln befolgt werden. Punkt vor Strich sowie die Beachtung von Klammern. Dabei kann ein Ausdruck beliebig Komplex sein. Eine Rechersche nach Ansätzen führte zu Stackoverflow. (Siehe  
\footnote{Abr. 11.03.2019 \url{https://stackoverflow.com/questions/8706356/boolean-expression-grammar-parser-in-c/8707598\#8707598}}
 und  \footnote{Abr. 11.03.2019 \url{http://coliru.stacked-crooked.com/a/c40382620fb75b75}}) Dieser Ansatz löste genau das Problem und wurde somit in das Projekt übernommen. 
%
\subsection{Automatische Prüfung von Abhängigkeiten}
Ein Problem was sich mit dem Einbinden der Lösung zum Parsen der Logikausdrücke ergab, war die Abhängigkeit zu Boost. Auf dem Desktop Computer auf dem die Lösung getestet wurde, konnte das Projekt dank installiertem boost Paket ohne Probleme übersetetzt werden. Da es sich bei dem  Zielsystem jedoch um eine ARM Architektur handelt, musste der code dort Übersetzt werden. In den Paketquellen des dort installieren Raspian, ist jedoch eine ältere Version des Boost Pakets hinterlegt, was dazu führt dass Boost manuell heruntergeladen und gebaut werden muss. Obwohl sich dieses Problem im Laufe der Zeit durch aktualisierung des Paketes in den Paketquellen von Raspian von selbst lösen wird, muss dennoch geprüft werden ob die installierte Version den Ansprüchen genügt. Hierfür wurde ein Bash script erstellt, welches auch alle weiteren abhängigkeiten Prüft und gegebenenfalls installiert. Dazu zählen auch die in *REF* erwähnten Bibliotheken um die Hardware anzusprechen und wie in *REF* erwähnt der verwendete Kompiler. 
 
\subsection{Benamungsschnema und Klassenstruktur}
Nachdem die Auswertung beziehungsweise das Parsing der Logikausdrücke funktionierte, sollte auch die Auswertung der Bezeichner automatisiert werden. Ein Benmungsschema wurde dabei schon vorher erdacht. Es besteht aus einem führenden Großbuchstaben, gefolgt von einem Kleinbuchstaben und einer Zahl. Dabei wird der führende Buchstabe als Kanal bezeichnet, der zweite als Entität und die Ziffer als Pin-Nummer. So könnte zum Beispiel der Buchstabe \chphl{H} den Kanal Hardware beschreiben, welcher  wiederrum die Entitäten \chphl{i} für Input und \chphl{o} besitzt, welche jeweils ein Byte, also 8 Bits oder Pins haben. Im Programm repräsentiert eine Klasse mit dem Namen Channel die Kanäle, wobei jeder Kanal als Eigenschaft einen Vektor mit Entitäten führt. Eine Entität widerrum weiß, wie groß ihre Breite ist, also wie viele Bits sie hat und ob sie nur Lesbar oder auch schreibbar ist. Durch Überladung der Operatoren der beiden Klassen ist ein Zugriff im Format chnl['H']['i'] möglich. Die Entitäten erben von der Basisklasse Entity. Sie überschreiben Methoden um lesend oder schreibend, entweder auf den gesamten Inhalt auf einmal oder Bitweise auf den Inhalt zuzugreifen. Dafür sind die methoden read\_pin, write\_pin beziehungsweise read\_all und write\_all vorgesehen. 

\subsection{Laden der Programmlogik aus Datei}
Das Logikprogramm wurde im nächsten Schritt in eine Textdatei ausgelagert. Sie soll künftig von der grafischen Benutzeroberfläche automatisch erstellt werden können, kann aber nach wie vor auch von Hand erstellt werden. Die Datei enthält einen Ausgang pro Zeile, gefolgt von einem Gleichheitszeichen und den Abhängigkeiten. Zeichen die auf ein Semikolon folgen, werden dabei als Kommentar gewertet und ausgelassen. Sie wird bei Programmstart eingelesen und verbleibt dann im Speicher.  
\subsubsection{Erneutes Laden der Logik zur Laufzeit}
Zum erneuten einlesen der Logik, wurde ein Signalhandler vorgesehen, welcher auf das Signal SIGUSR1 hört. Die entsprechende Funktion ließt dann die Logikdaten erneut ein und überschreibt damit die vorherige version im Speicher. Damit muss die Steuerung nicht neu gestartet werden und ermöglicht das Steuerungsprogramm zur Laufzeit zu verändern.
\subsection{Interrupt}
\subsection{Merker Bausteine}
\subsection{Timer Bausteine}
\subsection{WebSockets und Virtuelle Kanäle}


Beschreibung, Begründung

