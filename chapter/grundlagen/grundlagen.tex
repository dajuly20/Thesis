\section{Grundlagen}\label{kap:grund}

%\subsubsection{Motivation}
In dieser Bachelor Thesis soll eine Speicherprogrammierbare Steuerung mittels eines Rasperry Pi nachempfunden werden. Der Fokus liegt dabei darauf, eine möglichst günstige Lösung zu schaffen, um auch Einsteigern die Möglichkeit zu bieten einfache Steuerungsprojekte zu realisieren. Nachfolgend wird kurz darauf eingegangen was die eine Speicherprogrammierbare Steuerung eigentlich ist und weshalb man sie benötigt. 
\subsubsection{Was ist eine SPS}
Die Grundsätzliche Funktion einer Speicherprogrammierbaren Steuerung ist, die Ermittlung der Ausgangswerte bzw. Schalterstellung durch eine logische Verknüpfung der Eingangswerte. Im einfachsten Beispiel, könnte ein an einen Eingang angeschlossener Schalter als Sensor dienen. Als Aktor könnte eine Leuchte dienen. Der Benutzer der Steuerung muss nun durch eine Logik für jeden Ausgang festlegen, in welchen Fällen dieser Ausgang aktiv sein soll. Doch wieso schließt man dann nicht einfach die Leuchte direkt an den Schalter an? Dies wäre bei einer einfachen Lampensteuerung sicherlich zu bevorzugen, jedoch handelt es sich bei den Szenarien die mit einer solchen Steuerung relisiert werden für Gewöhnlich um deutlich Komplexere Verschaltungen. Bei der klassischen installation für eine Torsteuerung Beispielsweise, wären mehrere Elektromechanische Relais, auch Schütze genannt nötig. Zudem bedürfte ein automatisches schließen des Tores ein Zeitrelais. Der Verdahtungsaufwand und Platzbedarf wären relativ hoch. Führt man stattdessen jedoch alle benötigten Sensoren auf eine Speicherprogrammierbare Steuerung wird der Verdrahtungsaufwand erheblich reduziert, was zu einer höheren Übersichtlichkeit führt und weniger Potential für Fehler bietet. Auch zieht eine Änderung im Logischen verhalten der Steuerung dann für gewöhnloch keinerlei Verdrahtungsänderungen mehr nach sich. Zuletzt sind auch die Kosten für Speicherprogrammierbare Steuerungen inzwischen auf einem Nivaeu, was klassische Steuerungen schnell unwirtschaftlich macht. 


\subsection{Konzept?}
\subsection{Ausgangssituation} 

Als Vorbild für dieses Projekt dient die Kleinsteuerung Easy vom Hersteller Eaton. Das Einstiegsmodeell bietet hier acht Eingänge und vier Ausgänge. Das Logikprogramm, welches die Eingänge der Steuerung logisch mit den Ausgängen verbindet wird hier, auf einem kleinen Display, direkt am Gerät erstellt. Dabei stehen neben den physikalischen Ein- und Ausgängen auch Zeitfunktionen oder Zählerbausteine zur Verfügung. *Erweiterbar* Im Programmiermodus wird links einen Pluspol und rechts einen Minuspol Symbolisiert. Der anzusteuernde Ausgang, welcher obligatorisch ist, steht dabei stehts ganz rechts. Der Stompfad kann nunmehr bis zum Pluspol durchgezeichnet werden, oder aber durch Sensoren unterbrochen und verzweigt werden. Aus diesem Schaltplan werden dann die boolschen Gleichungen gewonnen, die die Steuerung im Betrieb durchläuft um die Werte der Ausgänge zu bestimmen.
\subsubsection{Bedienoberfläche} 
Eine ähnliche Vorgehensweise ist auch in diesem Projekt geplant. Da der Rasperry Pi Netzwerkfähig ist, wurde jedoch anstatt einem Display am Gerät eine Bedienoberfläche gewählt, welche im Internetbroweser bedienbar ist. Als Basis für die Programmieroberfläche, wurde das Projekt CircuitVerse *REF* herangezogen. Hierbei handelt es sich um einen Logiksimulator, in welchem komplexe Logikschaltungen durch Drag\&Drop erstellt werden können. Das Quelloffene Projekt ist auf GitHub *REF* verfügbar und dank  der MIT *REF* Lizenz zur Erweiterung und Modifikation freigegeben. Dabei musste das Projekt vor allem durch eine Funktion ergänzt werden, um die Erstellte Logik in einem Format zu Exportieren, welche vom Backend verstanden wird. Weiterhin musste die zur Verfügung stehenden Ein- und Ausgänge dahingehend modifziert werden, dass nur Schaltungen erstellt werden können, die auch vom Backend verstanden werden. 
Im Vorbild kann der Schaltplan auch Laufzeitinformationen wiedergeben. So wird ein (symbolisch) unter Spannung stehender Zweig als breite Linie dargestellt, wärend unbestromte zweige schmal gezeichnet werden. Dies Laufzeitinformationen sollen in dieser Bachelorarbeit ebenfalls dargestellt werden.
\subsubsection{Backend} 
Als Schnittstelle zwischen Hardware und Bedienoberfläche wird eine Software eingesetzt, die das vom Benutzer erstellte Logikprogramm kontinuierlich durchläuft, und somit sicherstellt, das eine Änderung an einem Eingang, der Ablauf eines Timers etc. die Werte der davon abhängigen Ausgänge entsprechend verändert. Wie dies im Vorbild der Easy Kleinsteuerung gelöst wird, bestand kein Einblick.  

\subsubsection{Hardware} 
Wie schon vorab beschrieben, bildet ein Rasperry Pi die Grundlage für dieses Projekt. Dieser bietet von Haus aus einige GPIOs, welche  dazu  verwendet werden können um Sensoren abzugragen oder um Aktoren anzusteuern. Jedoch viel die Entscheigung darauf, eine Erweiterungskarte (HAT) zu diesem Zweck einzusetzen. Dies dient erstmals zum Schutz des Rasperry Pi, zudem bietet das eingesetzte Board jedoch Leuchtdioden an den Ausgängen, sowie Taster an den Eingängen, was das Testen erheblich vereinfacht. Da der Anschluss per SPI erfolgt, können theoretisch mehrere solcher Boards parralell betrieben werden. 

\subsection{Versionsverwaltung}
Da im bisherigen Studium lediglich auf SVN als Versionsverwaltung tiefer eingegangen wurde wobei eine Versionsverwaltung für ein Projekt dieser Größe als notwendig erachtet wurde, fiel der Gedanke auf GIT. Git ist eine dezentrale Versionsverwaltung, die notwendigkeit für einen Versionierungsserver entfällt hierdurch. Jedoch ist es in Git möglich ein- oder mehrere sogenannte Remotes hinzuzufügen. Das sind entfernte Git-Repositorys die mit dem lokalen repository synchronisiert werden können. In der Praxis wird Git häufig eingesetzt, weshalb sich diese Bachelorarbeit als einarbeitung anbot. Zunächst wurde ein lokales Git Repository erstellt, welches in Folge mit einem Remote-repository auf GitHub verbunden wurde. Angedacht war ein Development-Branch und jeweils ein Feature Branch welcher nach vollendung des entsprechenden Features in den Development Branch zurückgeführt werden soll. Darüber hinaus, soll ein Tagging erfolgen. Dabei soll für jeden Zeitblanken *WORD* im, in der vorhergehenden Projektplanung erstellten Gantt Diagramm, ein Tag erstellt werden. 


    
\clearpage
