 \section{Einleitung}
 Speicherprogrammierbare Steuerungen oder kurz SPS tauchen überall dort auf, wo große elektrische Maschinen eingesetzt werden. Dies ist vor allem in der Industrie der Fall. Ihr kleiner Bruder ist die Kleinsteuerung. Sie bietet die selben Kernfunktionen, hat jedoch eine deutlich kleinere Anzahl an Ein- und Ausgängen. Sie werden häufig von Elektroinstallateuren eingesetzt, wenn eine klassische Verbindungsprogrammierte Steuerung zum Beispiel durch Drahtbruch oder defekte Spulen in eingesetzten Relais nicht mehr korrekt funktionieren. Der Hersteller Eaton hat mit seinem Produkt \chphl{Easy} genau diese Zielgruppe im Blick. Die Programmierung erfolgt hier, als würde man klassische Schütz-Kontakte in Reihe schalten. Die Einstiegsgeräte sind relativ preiswert, doch kauft man sich in eine proprietäre Produktwelt ein, welche aufgrund von inkompatiblen Bauteilen und Bussystemen schwer wieder zu verlassen ist. So gestaltet sich die Erweiterung einer bestehenden Steuerung um die Möglichkeit einer Fernabfrage übers Internet als nahezu unmöglich, oder setzt den Austausch der kompletten Steuerung voraus. Dabei sind Ein- und Ausgänge doch eigentlich das selbe wie an jedem Raspberry Pi vorhandene GPIOs. Auf Basis dieser Überlegung und günstigen Preisen hierfür, entstand die Idee eine Lösung mittels Raspberry Pi zu erarbeiten. 
\subsection{Ziel der Arbeit}
Ziel dieser Arbeit ist es dabei vor allem eine möglichst günstige Möglichkeit zu schaffen um eine Steuerung zu realisieren, welche intuitiv programmiert werden kann und die Grundsätzliche Funktion der vorher eingesetzten Easy Steuerung um Funktionen zur Fernabfrage übers Internet und weitere Funktionen erweitert. Dabei soll das Projekt mittels Git versioniert werden um es anderen Entwicklern auf GitHub als Open-Source Software zur Verfügung zu stellen. Dabei wird das Projekt als MIT Lizensiert, was eine Modifikation sowie private und gewerbliche Nutzung und Verbreitung ausdrücklich gestattet. 
\subsection{Verwandte Arbeiten}
Weitere Projekte mit denen ähnliches möglich ist, sind hierbei das Kommerzielle Projekt Codesys *REF*. Auch zu nennen ist das Projekt Open-PLC *REF*   
\subsection{Aufbau der Arbeit}
\todo{ Aufbau der Arbeit fertigstellen}
\clearpage
