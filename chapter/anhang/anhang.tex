\section{Anhang}
\subsection*{Inhalt der beigelegten CD}
\begin{figure}[H]

\dirtree{%
.1 \textbackslash\textcolor{folder}{CD}.
.2 \textcolor{folder}{bla/}\DTcomment{Git-Auszug des Blah}.
.2 \textcolor{folder}{llvm/}\DTcomment{Git-Auszug des Bahg}.
.2 \textcolor{folder}{MC6809/}\DTcomment{Git-Auszug des Bloh}.
.2 talk{.}pdf\DTcomment{Im Rahmen der Arbeit gehaltener Vortrag}.
.2 thesis{.}pdf\DTcomment{Dieses Dokument}.
}
\caption{Inhalt der beigelegten CD}
\label{fig:cd}
\end{figure}

\subsection*{Eingesetzte Software}

In \autoref{tab:usedSoftware} werden die wichtigsten Programme, die zum Entwickeln verwendet wurden, aufgelistet.

\begin{table}[H]
\centering
\caption{Liste der eingesetzten Software}
\label{tab:usedSoftware}
\begin{tabular}{ll}
\hline
\todo{Look for Boost install script on RPI 3}
Programm       & Version      \\ \hline
CMake          & 3.13.4       \\
GCC            & 8.1.0        \\
Boost		   & 0.0.0		  \\
glibc          & 2.28         \\
libstdc++      & 3.4.25       \\
Linux Kernel   & 4.19.0       \\
LLVM           & 7.0.0        \\
Ninja          & 1.8.2        \\
Python2        & 2.7.15       \\
Python3        & 3.7.1        \\ \hline

\end{tabular}
\end{table}

