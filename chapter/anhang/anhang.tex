\section{Anhang}\label{chp:anhang}
\subsection*{Inhalt der beigelegten DVD}
\begin{figure}[H]

\dirtree{%
.1 \textbackslash\textcolor{folder}{DVD}.
.2 \textcolor{folder}{ControlPi/}\DTcomment{Git-Auszug des Projektordners}.
.2 \textcolor{folder}{Demo-Videos/}\DTcomment{Videos zur Funktionsweise}.
.2 \textcolor{folder}{libmcp23s17/}\DTcomment{Git-Auszug der Bibliothek Libmcp23s17}.
.2 \textcolor{folder}{libpifacedigitalcpp/}\DTcomment{Git-Auszug der Bibliothek Libpifacedigitalcpp}.
.2 ControlPi-Raspian-Strech{.}tar{.}gz\DTcomment{Komplettes System-Image}.
.2 talk{.}pdf\DTcomment{Im Rahmen der Arbeit gehaltener Vortrag}.
.2 thesis{.}pdf\DTcomment{Dieses Dokument}.
}
\caption{Inhalt der beigelegten CD}
\label{fig:cd}
\end{figure}

\subsection*{Eingesetzte Software}

In \autoref{tab:usedSoftware} werden die wichtigsten Programme, die zum Entwickeln verwendet wurden, aufgelistet.

\begin{table}[H]
\centering
\caption{Liste der eingesetzten Software}
\label{tab:usedSoftware}
\begin{tabular}{ll}
\hline
Programm       & Version      \\ \hline
CMake          & 3.13.4       \\
GCC            & 8.1.0        \\
Boost		   & 1.6.9		  \\
Boost Beast	   & 2.3.5		  \\
libmcp23s17    & 0.0.0		  \\
libpifacedigitalcpp & 0.0.0  \\
Linux Kernel   & 4.14.98      \\
Raspian 	   & 9 (strech)   \\
Apache 		   & 2.4.25       \\
PHP			   & 7.0.33		  \\

\end{tabular}
\end{table}

